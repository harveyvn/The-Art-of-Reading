\documentclass[10pt,a4paper]{report}
%
%
%% IF YOU EXPERIENCE ANY PROBLEM WITH THIS TEMPLATE CONTACT DR. ALESSIO GAMBI
%
%
\usepackage[a4paper, total={6in, 10in}]{geometry}

\usepackage{titling}
\usepackage[utf8]{inputenc}
\usepackage{csquotes}

%%%% Machinery to draw the rating stars
\usepackage{tikz}
\usetikzlibrary{shapes.geometric}
\newcommand{\Stars}[2][fill=yellow,draw=orange]{\begin{tikzpicture}[baseline=-0.35em,#1]
\foreach \X in {1,...,5}
{\pgfmathsetmacro{\xfill}{min(1,max(1+#2-\X,0))}
\path (\X*1.1em,0) 
node[star,draw,star point height=0.25em,minimum size=1em,inner sep=0pt,
path picture={\fill (path picture bounding box.south west) 
rectangle  ([xshift=\xfill*1em]path picture bounding box.north west);}]{};
}
\end{tikzpicture}}
%%%% Machinery to draw the rating stars

\usepackage{fancyhdr}
\thispagestyle{fancy}
\pagestyle{fancy}

\usepackage{paralist}

\usepackage{titlesec}
\titleformat{\section}{\normalfont\fontsize{12}{15}\bfseries}{\thesection}{1em}{}

\usepackage[backend=biber,style=numeric]{biblatex}
\addbibresource{biblio.bib}

\usepackage[english]{babel}
\usepackage{blindtext}

\renewcommand{\thesection}{\arabic{section}}

%%%% Related Work environments
\newcounter{RelatedWorkCounter}
\addtocounter{RelatedWorkCounter}{1}
\newcommand{\relatedwork}[3]{%
\paragraph{Paper:}\fullcite{#1}
\begin{compactdesc}
\item[- How:] #2
\item[- Why:] #3
\end{compactdesc}
\stepcounter{RelatedWorkCounter}
}

%%%% Critical question environments
\newcounter{QuestionCounter}
\addtocounter{QuestionCounter}{1}
\makeatletter
\newcommand{\criticalquestion}[1]{\def\criticalquestion@required{#1}\criticalquestion@opt}
\newcommand{\criticalquestion@opt}[1]{%
\paragraph{Q\theQuestionCounter: \criticalquestion@required}
#1%
\stepcounter{QuestionCounter}
}
\makeatother

%%%%%%%%%%%%%%%%%%%%%%%%%%%%%%%%%%%%%%%%%%
% Meta Data:
%%%%%%%%%%%%%%%%%%%%%%%%%%%%%%%%%%%%%%%%%%

\lhead{Vuong Nguyen}
\rhead{Topic: DeepX/Paper 03}
\title{DeepBillboard: Systematic Physical-World Testing of Autonomous Driving Systems}

\begin{document}
\begin{center}
\textbf{\thetitle}
\end{center}

%%%%%%%%%
% If your text is too long and you need to choose what part to cut down between Summary and Critical Discussion, always cut down on the Summary! We all have read the paper, so the really interesting part is your opinion!
% follow the lane
% regain its center on the lane
% drive within a lane center
% avoid land departure
% to depart the road
% to move away from the center of the lane
% cause the self-driving car to break out of the land bounds
% cause the ego-car to drive away from the lane center
% leads the ego-car to drive out of the road

% be amenable for
% be responsible for
%%%%%%%%%
\section{Summary}
%1/2 Page
The paper discusses the advantages of Deep Neural Networks (DNNs) in the testing of autonomous vehicles and an existing critical missing aspect of the autonomous driving testing domain. 
%
Thus, it proposes a systematic physical-world testing approach, namely DeepBillboard to produce an adversarial billboard test, which supposes to work under various driving conditions and cause a missteering to targeted steering model.
%
The proposed approach is to generate a printable billboard image with perturbations that aims to enhance test effectiveness and exploit the influence of digital and physical perturbation generation for autonomous systems' steering decisions.

In this paper, the DeepBillboard focused on driving by roadside billboards when it generates adversarial billboard images along a road that tries to mislead the steering angle or cause off-tracking.
%
In addition, the approach consists of the CNN-based steering models which will be responsible for receiving images captured by multiple sensors as inputs and then generating steering angle decisions as outputs.
%
The authors defined the Joint Loss Optimization algorithm for producing several adversarial perturbations that will result in misleading steering decisions of an autonomous vehicle with various driving conditions.
%
Furthermore, DeepBillboard provided implementations to not only address the overlapped perturbations issues among multiple frames to avoid interferences among those frames but also define Non-printability Score (NPS) to enhance perturbation in the physical world.
%
For different environmental conditions, the paper introduced a color adjustment function that adjusts color differences in the video captured by a dashcam which intends to minimize impacts on the adversarial efficacy.
%
To indicate the effectiveness, the authors evaluate the effectiveness of their algorithm by measuring the Average Angle Errors (AAE) of all frames digitally and physically.
%
As a result, DeepBillboard made all steering models generate an observable average steering angle and the steering angle errors increase as the billboard size increases.
%
Regarding Physical Case Study, the steering angles were straight with the white billboard while it turned to a certain degree with the bright or dark billboard.
%
Moreover, DeepBillboard represented the large Exp\_AAE and Test\_AAE values of test effectiveness under different weather conditions with slight differences between those metrics which supposes to indicate the approach’s effectiveness in the physical world setting.


\section{Critical Discussion}
%1/2 Page
First of all, the paper was well written. In the beginning, it not only identified problems that the DNN testing framework faced in the automotive domain but also proposed an alternative solution to address those problems transparently.
%
Furthermore, researchers attempted to make their paper approachable to the public by explaining their proposed concepts such as Evaluating Matrices, the design of DeepBillboard, and the Joint Loss Optimization algorithm, etc. in detail. 
%
Therefore, it makes those terms comprehensible, even for people with minimal software testing experience.
%
Additionally, the authors have not only discussed challenges relating to physical attacks and issues relating to the design of DeepBillboard such as Overlapped Perturbations or Color Difference under Various Environment Conditions but also provided comprehensive solutions to address those issues successfully.
%
Thus, the audience can notice other aspects of the approach and acquire new insights from those solutions.
%
Finally, the goals of the experiment are identified and validated carefully by the authors.
%
The authors propose and explain new evaluation metrics and methodology to measure the test effectiveness of perturbations and then measure the efficacy of DeepBillboard in various digital and physical case studies.
%
Moreover, the configuration of those empirical studies has been analyzed deeply and the authors also provided compelling evidence with proper statistical figures and tables to support their arguments in most of the given case studies.
%

On the other hand, it is ideally advised to introduce statistical significance such as $p$-value to measure the certainty of the results to increase the final conclusion's reliability.
%
In addition, from a professional standpoint, it is ideally advised to introduce a research goal and research questions.
%
Last but not least, experiment settings are limited in the discussion about the influence of a single billboard on autonomous vehicles.
%
It would be better if the paper can demonstrate the empirical studies to explore the contribution of other vehicles with a printable billboard to the missteering of the autonomous vehicle during the driving process in the future.
%
To conclude, I’m almost convinced that the DeepBillboard is by far an ideal, pragmatic approach for generating the effective driving adversarial scenarios, which can expose many safety-critical problems of the self-driving software and the proposed approach might be sufficient to translate this concept into applicable use.

\newpage 
% Paper rating, critical questions and related work sections must always appear on the second page of the summary

\section{Rating}
% Add here the overall rating of the paper (1 start is BAD, 5 starts is VERY GOOD). Please explain in one or two sentences the reason of your evaluation and whether you suggests the paper for the next edition of the seminar.
\Stars{5}

The paper has defined concepts transparently and illustrated the empirical evaluation carefully with proper arguments and statistical methods. Even though, there are still minor problems, they can be solved in future work, so I would rate this paper 5 stars.

Of course I highly recommend this paper for the next semester !

\section{Critical Questions}
% A least 2 questions here. If possible try to answer them or write down 
\criticalquestion{The approach is limited to the scenario including a single roadside billboard. If we inject more than one adversarial billboards along the roadside at the same time, how does it significantly affect the steering angle decision more than a single adversarial billboard case?} 
% {..}

\criticalquestion{If we introduce more than one vehicles along with a roadside billboard, are there any negative or positive effects to DeepBillboard's efficacy evaluation?}% Optional answer follow
% {...}

\section{Related Work}
% Remember that you MUST list at least 4 related work here ! Fill the bib file will all the required information and build your bibliography before submitting the paper !
\paragraph{How many other papers did you considered during for the related work?}
4


% FIRST
\relatedwork%
% Put the citation key corresponding to the paper you selected here:
{wiyatno2019physical}
% Explain how did you find the paper here (check the slides to see how you can effectively find related work papers)
{Searched in Google Scholar as a citation paper of the main paper.}
% Explain why this paper is related here (do not just say, it has the same content or a similar title...)
{The paper presents a similar method to the main paper by creating adversarial billboards that caused a victim model misleading and adding perturbation to whole sequential track rather than a single frame.}

% SECOND
\relatedwork
%% Put the cite key corresponding to the paper here:
{kong2019generating}
%% Explain how did you find the paper here:
{Searched in Google Scholar as a citation paper of the main paper.}
%% Explain why this paper is related here:
{The paper proposed GAN-based framework PhysGAN to generates physical-world-resilient adversarial examples for misleading autonomous driving systems in a continuous manner.}

% THIRD
\relatedwork
%% Put the cite key corresponding to the paper here:
{zhang2020machine}
%% Explain how did you find the paper here:
{Searched in Google Scholar as a citation paper of the main paper.}
%% Explain why this paper is related here:
{Vaguely related but the paper covers a comprehensive survey of Machine Learning Testing research including 138 papers and it classifies the main paper to the Domain-specific Test Input Synthesis.}

% FOURTH - AND PROBABLY LAST 
\relatedwork
%% Put the cite key corresponding to the paper here:
{eykholt2018robust}
%% Explain how did you find the paper here:
{Searched in Google Scholar with keywords Physical-world Testing, Autonomous driving systems.}
%% Explain why this paper is related here:
{The paper proposes a new attack approach which shares the concept of Join Optimization algorithm. The approach is called Robust Physical Perturbations that generates perturbations by taking images under different conditions into account.}


\end{document}