\documentclass[10pt,a4paper]{report}
%
%
%% IF YOU EXPERIENCE ANY PROBLEM WITH THIS TEMPLATE CONTACT DR. ALESSIO GAMBI
%
%
\usepackage[a4paper, total={6in, 10in}]{geometry}

\usepackage{titling}
\usepackage[utf8]{inputenc}

%%%% Machinery to draw the rating stars
\usepackage{tikz}
\usetikzlibrary{shapes.geometric}
\newcommand{\Stars}[2][fill=yellow,draw=orange]{\begin{tikzpicture}[baseline=-0.35em,#1]
\foreach \X in {1,...,5}
{\pgfmathsetmacro{\xfill}{min(1,max(1+#2-\X,0))}
\path (\X*1.1em,0) 
node[star,draw,star point height=0.25em,minimum size=1em,inner sep=0pt,
path picture={\fill (path picture bounding box.south west) 
rectangle  ([xshift=\xfill*1em]path picture bounding box.north west);}]{};
}
\end{tikzpicture}}
%%%% Machinery to draw the rating stars

\usepackage{fancyhdr}
\thispagestyle{fancy}
\pagestyle{fancy}

\usepackage{paralist}

\usepackage{titlesec}
\titleformat{\section}{\normalfont\fontsize{12}{15}\bfseries}{\thesection}{1em}{}

\usepackage[backend=bibtex8,style=numeric]{biblatex}
\addbibresource{biblio.bib}

\usepackage[english]{babel}
\usepackage{blindtext}

\renewcommand{\thesection}{\arabic{section}}

%%%% Related Work environments
\newcounter{RelatedWorkCounter}
\addtocounter{RelatedWorkCounter}{1}
\newcommand{\relatedwork}[3]{%
\paragraph{Paper:}\fullcite{#1}
\begin{compactdesc}
\item[- How:] #2
\item[- Why:] #3
\end{compactdesc}
\stepcounter{RelatedWorkCounter}
}

%%%% Critical question environments
\newcounter{QuestionCounter}
\addtocounter{QuestionCounter}{1}
\makeatletter
\newcommand{\criticalquestion}[1]{\def\criticalquestion@required{#1}\criticalquestion@opt}
\newcommand{\criticalquestion@opt}[1]{%
\paragraph{Q\theQuestionCounter: \criticalquestion@required}
#1%
\stepcounter{QuestionCounter}
}
\makeatother

%%%%%%%%%%%%%%%%%%%%%%%%%%%%%%%%%%%%%%%%%%
% Meta Data:
%%%%%%%%%%%%%%%%%%%%%%%%%%%%%%%%%%%%%%%%%%

\lhead{Vuong Nguyen}
\rhead{Topic: Basic Search/Paper 02}
\title{Automatically Testing Self-Driving Cars with Search-Based Procedural Content Generation}

\begin{document}
\begin{center}
\textbf{\thetitle}
\end{center}

%%%%%%%%%
% If your text is too long and you need to choose what part to cut down between Summary and Critical Discussion, always cut down on the Summary! We all have read the paper, so the really interesting part is your opinion!
% follow the lane
% regain its center on the lane
% drive within a lane center
% avoid land departure
% to depart the road
% to move away from the center of the lane
% cause the self-driving car to break out of the land bounds
% cause the ego-car to drive away from the lane center
% leads the ego-car to drive out of the road

% be amenable for
% be responsible for
%%%%%%%%%
\section{Summary}
%1/2 Page
The paper discusses the testing challenges of autonomous vehicles in real life situations since  such testing settings are ineffective and dangerous. 
%
Despite being computer-intensive, virtual environments are still in use as an alternative to the aforementioned approach. And thus, the authors proposed a search-based procedural content generation approach along with its AsFAULT prototype to address these problems. 
%
The paper introduces the Procedural Content Generation (PCG) as an algorithm to generate roads automatically and the Search-based Testing (SBST) as an algorithm to generate effective test cases in self-driving car contexts.
% 
By combining two algorithms, this approach will be able to produce efiicient test cases for exposing problems of self-driving car software systems.

Despite the fact that AsFAULT’s road generation based on PCG is challenging due to complexities in geometrical properties for making their virtual roads realistic, 
this algorithm has still proven its ability not only in generating road networks precisely but also testing lane keeping systems effectively.
%
To elaborate, the algorithm initializes sequences of points and then generates valid road segments, which construct valid road networks.
%
Furthermore, AsFAULT utilizes a search-based approach with an aim to evolve their road networks.
%
In addition, the approach encodes road networks as a hierarchical data structure which allows AsFAULT to evolve test cases and to generate roads by itself.
%
AsFAULT also provides navigation, in which driving tasks can conduct and those paths are modeled as a graphic representation.
%
In order to indicate that the AsFAULT is superior in comparison to Random Search, researchers formulated four research questions and concluded empirical studies on two different self-driving software systems such as BeamNG.AI and DeepDriving.
%
With regards to evaluation metrics, they performed Mann-Whitney U-test \textit{p}-values for significance tests and Vargha–Delaney statistics to measure the effective size.
%
As a result, the approach demonstrated its capability by finding the number of \textit{OBE} in both test subjects which are twice as many \textit{OBE} as random search. It also indicates the difficulty in evolution of single-road networks in comparison with multi-road network as well as illustrates that although small maps can speed up the searching, the large maps can increase effectiveness to test cases.


\section{Critical Discussion}
%1/2 Page
First of all, the paper was well written. In the abstract of the paper, it not only identified problems that the traditional testing framework faced in self-driving cars but also proposed an alternative solution to address those problems successfully. In the same way, the paper provided compelling evidence to support their arguments on why their approach statistically outperformed Random Search in most of the given case studies. 
%
Furthermore, researchers attempted to make their paper approachable to the public by explaining basic concepts such as naturalistic field operational tests, PCG mechanism and SBST for Lane Keeping in detail. Therefore, it makes those terms comprehensible. 
%
Finally, the goals of the experiment are identified clearly and measured carefully by the authors. To elaborate, the configuration of a search-based testing has been analyzed deeply and the authors also discussed several possible drawback such as a large number of possible generating roads or paths causing high pressure to the software, etc. even if some have already been solved.
%
In addition, researchers translated their research into an applicable use through empirical studies on BeamNG.AI and DeepDriving. The results then were measured by evaluation metrics such as Mann-Whitney U-test \textit{p}-values and Vargha–Delaney statistic, so that they are are reliable and adequate to justify the hypothesis.

On the other hand, it might be possible that the measures used are not comparable across experiments. Thus, the authors should propose the solution on how to reduce a construct validity threat. 
%
In addition, the paper introduced DeepDriving extension which adjust the prediction quality of DeepDriving and if the predictions are not suitable, the control will switch to BeamNG.AI which might cause a potential bias in the experiment.
. 
%
Last but not least, experiment settings are limited in fixed width and single lane per traffic direction due to its simplification and fast generation, however it will be better if the AsFAULT can solve the problem with multi-lane per traffic direction in the future work since these kind of streets are quite popular around the world.

To conclude, I’m almost convinced that the combination of PCG and SBST is by far an ideal, pragmatic approach for generating the effective road networks, which can recognize many safety-critical problems of the self-driving car.
\newpage 
% Paper rating, critical questions and related work sections must always appear on the second page of the summary

\section{Rating}
% Add here the overall rating of the paper (1 start is BAD, 5 starts is VERY GOOD). Please explain in one or two sentences the reason of your evaluation and whether you suggests the paper for the next edition of the seminar.
\Stars{4}

The paper has defined many  concepts transparently and illustrated the empirical evaluation carefully with proper research questions and statistical methods. However, due to the minor limitations above, I would love to give this paper 4 stars.

Of course I highly recommend this paper for the next semester !

\section{Critical Questions}
% A least 2 questions here. If possible try to answer them or write down 
\criticalquestion{This is a very important question}% Optional answer follow
{Those are some important notes about the question or a possible answer to it.}

\criticalquestion{}% Optional answer follow
% {Those are some important notes about the question or a possible answer to it.}

\section{Related Work}
% Remember that you MUST list at least 4 related work here ! Fill the bib file will all the required information and build your bibliography before submitting the paper !
\paragraph{How many other papers did you considered during for the related work?}
FILL ME WITH AN ACCURATE NUMBER...


% FIRST
\relatedwork%
% Put the citation key corresponding to the paper you selected here:
{chen_deepdriving:_2015}%
% Explain how did you find the paper here (check the slides to see how you can effectively find related work papers)
{Keywords search using Google Scholar}
% Explain why this paper is related here (do not just say, it has the same content or a similar title...)
{Some basic explanation here.}

% SECOND
%\relatedwork%
%% Put the cite key corresponding to the paper here:
%{key1}%
%% Explain how did you find the paper here:
%{Keywords search using Google Scholar}
%% Explain why this paper is related here:
%{Some basic explanation here.}

% THIRD
%\relatedwork%
%% Put the cite key corresponding to the paper here:
%{key1}%
%% Explain how did you find the paper here:
%{Keywords search using Google Scholar}
%% Explain why this paper is related here:
%{Some basic explanation here.}

% FOURTH - AND PROBABLY LAST 
%\relatedwork%
%% Put the cite key corresponding to the paper here:
%{key1}%
%% Explain how did you find the paper here:
%{Keywords search using Google Scholar}
%% Explain why this paper is related here:
%{Some basic explanation here.}


\end{document}