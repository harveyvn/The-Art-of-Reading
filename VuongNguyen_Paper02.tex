\documentclass[10pt,a4paper]{report}
%
%
%% IF YOU EXPERIENCE ANY PROBLEM WITH THIS TEMPLATE CONTACT DR. ALESSIO GAMBI
%
%
\usepackage[a4paper, total={6in, 10in}]{geometry}

\usepackage{titling}
\usepackage[utf8]{inputenc}

%%%% Machinery to draw the rating stars
\usepackage{tikz}
\usetikzlibrary{shapes.geometric}
\newcommand{\Stars}[2][fill=yellow,draw=orange]{\begin{tikzpicture}[baseline=-0.35em,#1]
\foreach \X in {1,...,5}
{\pgfmathsetmacro{\xfill}{min(1,max(1+#2-\X,0))}
\path (\X*1.1em,0) 
node[star,draw,star point height=0.25em,minimum size=1em,inner sep=0pt,
path picture={\fill (path picture bounding box.south west) 
rectangle  ([xshift=\xfill*1em]path picture bounding box.north west);}]{};
}
\end{tikzpicture}}
%%%% Machinery to draw the rating stars

\usepackage{fancyhdr}
\thispagestyle{fancy}
\pagestyle{fancy}

\usepackage{paralist}

\usepackage{titlesec}
\titleformat{\section}{\normalfont\fontsize{12}{15}\bfseries}{\thesection}{1em}{}

\usepackage[backend=bibtex8,style=numeric]{biblatex}
\addbibresource{biblio.bib}

\usepackage[english]{babel}
\usepackage{blindtext}

\renewcommand{\thesection}{\arabic{section}}

%%%% Related Work environments
\newcounter{RelatedWorkCounter}
\addtocounter{RelatedWorkCounter}{1}
\newcommand{\relatedwork}[3]{%
\paragraph{Paper:}\fullcite{#1}
\begin{compactdesc}
\item[- How:] #2
\item[- Why:] #3
\end{compactdesc}
\stepcounter{RelatedWorkCounter}
}

%%%% Critical question environments
\newcounter{QuestionCounter}
\addtocounter{QuestionCounter}{1}
\makeatletter
\newcommand{\criticalquestion}[1]{\def\criticalquestion@required{#1}\criticalquestion@opt}
\newcommand{\criticalquestion@opt}[1]{%
\paragraph{Q\theQuestionCounter: \criticalquestion@required}
#1%
\stepcounter{QuestionCounter}
}
\makeatother

%%%%%%%%%%%%%%%%%%%%%%%%%%%%%%%%%%%%%%%%%%
% Meta Data:
%%%%%%%%%%%%%%%%%%%%%%%%%%%%%%%%%%%%%%%%%%

\lhead{Vuong Nguyen}
\rhead{Topic: Basic Search/Paper 02}
\title{Automatically Testing Self-Driving Cars with Search-Based Procedural Content Generation}

\begin{document}
\begin{center}
\textbf{\thetitle}
\end{center}

%%%%%%%%%
% If your text is too long and you need to choose what part to cut down between Summary and Critical Discussion, always cut down on the Summary! We all have read the paper, so the really interesting part is your opinion!
%follow the lane
%regain its center on the lane
%drive within a lane
%avoid land departure
%to depart the road
%to move away from the center of the lane
%%%%%%%%%
\section{Summary}
%1/2 Page
The paper discusses the testing challenges of autonomous vehicles due to danger in real traffic and computer intensive in virtual environment and then the authors propose a search-based procedural content generation approach to address these challenges. 
Virtual testing applies X-in-the-loop paradigm by using machine learning techniques to test self-driving car.
The approach consists of procedural content generation (PCG) - the algorithm to create a game content and search-based testing (SBST) - the algorithm to generate test cases in the self-driving car context with an aim to produce scenarios for exposing problems of self-driving car software systems. 
Despite the fact PCG for virtual road generation is challenging task due to complex geometrical properties for making their virtual roads closing to the realities, this algorithm has proven its potential ability to generate virtual road networks precisely and effectively.
To elaborate, the algorithm initializes virtual road as sequences of points - polylines for simplifying not only the rendering operation but also evaluation of geometric properties.
Furthermore, road generation mechanism only generates valid roads, which are not self-intersect and they do not cross the map boundary either. In addition, the front line of the former road segment will be the back line of the latter in order to ensure the gapless condition.
All of them contributes to construct valid road networks, where one valid road intersects to at least one road at the starting point IP of both roads.

A search-based approach to improve the network with an aim of making the autonomous cars to deviate from the road.
Finally, they evaluate their model on two self-driving car software systems and explored that AsFault does not only have a better performance in speed but also improve a test suite generation when it can catch up to twice as many as lane deviation in comparison with other random testing (...).




\section{Critical Discussion}
%1/2 Page
\blindtext
\blindtext
\blindtext
\blindtext

\newpage 
% Paper rating, critical questions and related work sections must always appear on the second page of the summary

\section{Rating}
% Add here the overall rating of the paper (1 start is BAD, 5 starts is VERY GOOD). Please explain in one or two sentences the reason of your evaluation and whether you suggests the paper for the next edition of the seminar.
\Stars{5}

Some text to explain why I gave 5 starts to this paper... 

Of course I would love to suggest this paper for the next semester !


\section{Critical Questions}
% A least 2 questions here. If possible try to answer them or write down 
\criticalquestion{This is a very important question}% Optional answer follow
{Those are some important notes about the question or a possible answer to it.}

\criticalquestion{This is another very important question}% Optional answer follow
{Those are some important notes about the question or a possible answer to it.}

\section{Related Work}
% Remember that you MUST list at least 4 related work here ! Fill the bib file will all the required information and build your bibliography before submitting the paper !
\paragraph{How many other papers did you considered during for the related work?}
FILL ME WITH AN ACCURATE NUMBER...


% FIRST
\relatedwork%
% Put the citation key corresponding to the paper you selected here:
{chen_deepdriving:_2015}%
% Explain how did you find the paper here (check the slides to see how you can effectively find related work papers)
{Keywords search using Google Scholar}
% Explain why this paper is related here (do not just say, it has the same content or a similar title...)
{Some basic explanation here.}

% SECOND
%\relatedwork%
%% Put the cite key corresponding to the paper here:
%{key1}%
%% Explain how did you find the paper here:
%{Keywords search using Google Scholar}
%% Explain why this paper is related here:
%{Some basic explanation here.}

% THIRD
%\relatedwork%
%% Put the cite key corresponding to the paper here:
%{key1}%
%% Explain how did you find the paper here:
%{Keywords search using Google Scholar}
%% Explain why this paper is related here:
%{Some basic explanation here.}

% FOURTH - AND PROBABLY LAST 
%\relatedwork%
%% Put the cite key corresponding to the paper here:
%{key1}%
%% Explain how did you find the paper here:
%{Keywords search using Google Scholar}
%% Explain why this paper is related here:
%{Some basic explanation here.}


\end{document}