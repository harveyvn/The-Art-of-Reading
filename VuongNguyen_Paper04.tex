\documentclass[10pt,a4paper]{report}
%
%
%% IF YOU EXPERIENCE ANY PROBLEM WITH THIS TEMPLATE CONTACT DR. ALESSIO GAMBI
%
%
\usepackage[a4paper, total={6in, 10in}]{geometry}

\usepackage{titling}
\usepackage[utf8]{inputenc}

%%%% Machinery to draw the rating stars
\usepackage{tikz}
\usetikzlibrary{shapes.geometric}
\newcommand{\Stars}[2][fill=yellow,draw=orange]{\begin{tikzpicture}[baseline=-0.35em,#1]
\foreach \X in {1,...,5}
{\pgfmathsetmacro{\xfill}{min(1,max(1+#2-\X,0))}
\path (\X*1.1em,0) 
node[star,draw,star point height=0.25em,minimum size=1em,inner sep=0pt,
path picture={\fill (path picture bounding box.south west) 
rectangle  ([xshift=\xfill*1em]path picture bounding box.north west);}]{};
}
\end{tikzpicture}}
%%%% Machinery to draw the rating stars

\usepackage{fancyhdr}
\thispagestyle{fancy}
\pagestyle{fancy}

\usepackage{paralist}

\usepackage{titlesec}
\titleformat{\section}{\normalfont\fontsize{12}{15}\bfseries}{\thesection}{1em}{}

\usepackage[backend=bibtex8,style=numeric]{biblatex}
\addbibresource{biblio.bib}

\usepackage[english]{babel}
\usepackage{blindtext}

\renewcommand{\thesection}{\arabic{section}}

%%%% Related Work environments
\newcounter{RelatedWorkCounter}
\addtocounter{RelatedWorkCounter}{1}
\newcommand{\relatedwork}[3]{%
\paragraph{Paper:}\fullcite{#1}
\begin{compactdesc}
\item[- How:] #2
\item[- Why:] #3
\end{compactdesc}
\stepcounter{RelatedWorkCounter}
}

%%%% Critical question environments
\newcounter{QuestionCounter}
\addtocounter{QuestionCounter}{1}
\makeatletter
\newcommand{\criticalquestion}[1]{\def\criticalquestion@required{#1}\criticalquestion@opt}
\newcommand{\criticalquestion@opt}[1]{%
\paragraph{Q\theQuestionCounter: \criticalquestion@required}
#1%
\stepcounter{QuestionCounter}
}
\makeatother

%%%%%%%%%%%%%%%%%%%%%%%%%%%%%%%%%%%%%%%%%%
% Meta Data:
%%%%%%%%%%%%%%%%%%%%%%%%%%%%%%%%%%%%%%%%%%

\lhead{Vuong Nguyen}
\rhead{Topic: Advanced Search/Paper 03}
\title{Testing Vision-Based Control Systems Using Learnable Evolutionary Algorithms}

\begin{document}
\begin{center}
\textbf{\thetitle}
\end{center}

%%%%%%%%%
% If your text is too long and you need to choose what part to cut down between Summary and Critical Discussion, always cut down on the Summary! We all have read the paper, so the really interesting part is your opinion!
% follow the lane
% regain its center on the lane
% drive within a lane center
% avoid land departure
% to depart the road
% to move away from the center of the lane
% cause the self-driving car to break out of the land bounds
% cause the ego-car to drive away from the lane center
% leads the ego-car to drive out of the road

% be amenable for
% be responsible for
%%%%%%%%%
\section{Summary}
%1/2 Page
The paper discusses the testing challenges of autonomous vehicles in unstructured and human shared environments since such testing settings exposes the low performance and mission achievement issues. 
%
Despite compute-intensive tasks, simulation-based testing is still in use as an alternative to the aforementioned approach because it avoids catastrophic accidents, explores different number of situations and controls input and output analysis.
%
And thus, the authors proposed a testing framework based on MORSE (Modular Open Robots Simulation Engine) to provide a new perspective toward future testing environments to stress the navigation service.
%

The approach consists of the use of a robotic simulator named $Mana$ and virtual worlds generated by Procedural Content Generation (PCG) algorithm. 
%
The researchers collected relevant environmental attributes and robotic characteristics to build a world model for navigation scenarios.
%
Then, Blender functions and predefined objects instantiates the world model as a simulation environment.
%
Regarding data collection and analysis, researchers gathered data which is relevant to the navigation tracking and data sources come from the point of view of the robot and the external observer respectively.
%
Each simulation run is sorted as a binomial classification task depending on the successful or failed state of the test.
%
To evaluate the practical effectiveness, researchers validated the research by addressing three research questions.
%
The first question debates an influence that control parameters have on level of difficulties (easy, challenging, and very difficult configuration).
%
The next question examines which control parameters impact the level of indeterminism while the final is to compare the evolution of the fault navigation service among different level of difficulties.
%
As a result, there is an obvious connection between the control parameters and levels of difficulty when the deformation $d$ increases, it indicates the general trend of the rising of difficult level.
%
Because of many small obstacles, the large number of obstacles and the bigger number of subdivisions of the ground surface, the level of difficulty also notices an increase.
%
Furthermore, researchers argue that any control parameters can affect to the evolution of diverse trajectories, not only the difficulty level as expected.
%
And finally, both the challenging and very difficult configuration reflect a higher revealing power than the easy configuration with respect to a faulty version of navigation.



\section{Critical Discussion}
%1/2 Page
First of all, the paper was well written. In the beginning, it not only identified problems that the traditional testing framework faced in self-driving cars but also proposed an alternative solution to address those problems successfully.
%
Furthermore, researchers attempted to make their paper persuasive since they provide compelling evidence to support their arguments on why their approach statistically outperformed a baseline algorithm by answering two research questions with experimental studies.  
%
Finally, the goals of the experiment are identified and measured carefully by the authors.
%
To elaborate, the description of objective measures such as Multi-objective search and Decision tree learning is transparent throughout the paper. 
%
The results then were measured by evaluation metrics such as Hypervolume (HV), Generational Distance (GD), and Spread (SP), the Wilcoxon Rank Sum test, and Vargha-Delaney, so that they are reliable and adequate to justify the hypothesis.
%

On the other hand, the papers organized interviews with few engineers to prove the approach's advantages but the number of people might not be substantial to claim the usefulness of the approach under the statistical point of view.
%
Besides, different configurations can lead to different result analysis. Therefore, the research should add more case studies besides AEB to describe the effects of critical behaviors and scenarios after changing typical parameters.
%
Also, the AEB system is one of many fundamental problems of Automotive Software Systems. 
%
The researchers have constructed this case study along with numerous parameter variables and complex protocol whereas they miss mentioning the ability to reuse those things on other problems in the same domain such as autonomous parking.
%
Therefore, it might be time-consuming and require lots of efforts in not only understanding the approach and learning how to apply it to a new problem.
%
Last but not least, the paper introduces a lot of concepts without proper correlation. 
%
Thus, it may lead to an increase in the complexity of the paper which makes it incomprehensible, especially for people with minimal software testing experience. 
%
To conclude, I’m interested in the combination of NSGA-II and Decision Tree Model is by far an ideal, pragmatic approach for generating effective critical test cases to expose problems of the self-driving car but it might be tough to translate the concept into applicable use.
\newpage 
% Paper rating, critical questions and related work sections must always appear on the second page of the summary

\section{Rating}
% Add here the overall rating of the paper (1 start is BAD, 5 starts is VERY GOOD). Please explain in one or two sentences the reason of your evaluation and whether you suggests the paper for the next edition of the seminar.
\Stars{5}

The paper has defined many  concepts transparently and illustrated the empirical evaluation carefully with proper research questions and statistical methods. Even though there are still small problems as mentioned above, so I would rate this paper 3 stars.

Of course I highly recommend this paper for the next semester !

\section{Critical Questions}
% A least 2 questions here. If possible try to answer them or write down 
\criticalquestion{What does it mean when authors argue using decision tree is better than SVM and other Machine Learning techniques due to its understandable boundaries? }% Optional answer follow
%{Those are some important notes about the question or a possible answer to it.}

\criticalquestion{The authors suggest an upper threshold to control the number of vectors in each tree leaf not below a certain lower threshold. How does it prevent overfitting?}% Optional answer follow
% {Those are some important notes about the question or a possible answer to it.}

\section{Related Work}
% Remember that you MUST list at least 4 related work here ! Fill the bib file will all the required information and build your bibliography before submitting the paper !
\paragraph{How many other papers did you considered during for the related work?}
4


% FIRST
\relatedwork%
% Put the citation key corresponding to the paper you selected here:
{chen2018beyond}
% Explain how did you find the paper here (check the slides to see how you can effectively find related work papers)
{Searched in Google Scholar with keywords Beyond Evolutionary algorithms and Search-based Software Engineering}
% Explain why this paper is related here (do not just say, it has the same content or a similar title...)
{The paper presents an approach to address computer-intensive problem of evolutionary methods by using fewer solutions.}

% SECOND
\relatedwork
%% Put the cite key corresponding to the paper here:
{campos2017empirical}
%% Explain how did you find the paper here:
{Searched in Google Scholar with keywords Evolutionary Algorithms and Test Suite Generation.}
%% Explain why this paper is related here:
{The paper demonstrates a comparison between evolutionary algorithms and random testing to find the most effective one for test generation.}

% THIRD
\relatedwork
%% Put the cite key corresponding to the paper here:
{buhler2004automatic}
%% Explain how did you find the paper here:
{Searched in Google Scholar with keywords Autonomous Parking and Evolutionary Algorithm.}
%% Explain why this paper is related here:
{The paper describes an algorithm that define suitable fitness functions to evaluate the quality of parking.}

% FOURTH - AND PROBABLY LAST 
\relatedwork
%% Put the cite key corresponding to the paper here:
{khosrowjerdi2017learning}
%% Explain how did you find the paper here:
{Searched in Google Scholar with keywords Critical Test Cases, Autonomous Vehicles.}
%% Explain why this paper is related here:
{The paper provides a method by applying machine learning and model-checking techniques for test case generation.}


\end{document}