\documentclass[10pt,a4paper]{report}
%
%
%% IF YOU EXPERIENCE ANY PROBLEM WITH THIS TEMPLATE CONTACT DR. ALESSIO GAMBI
%
%
\usepackage[a4paper, total={6in, 10in}]{geometry}

\usepackage{titling}
\usepackage[utf8]{inputenc}

%%%% Machinery to draw the rating stars
\usepackage{tikz}
\usetikzlibrary{shapes.geometric}
\newcommand{\Stars}[2][fill=yellow,draw=orange]{\begin{tikzpicture}[baseline=-0.35em,#1]
\foreach \X in {1,...,5}
{\pgfmathsetmacro{\xfill}{min(1,max(1+#2-\X,0))}
\path (\X*1.1em,0) 
node[star,draw,star point height=0.25em,minimum size=1em,inner sep=0pt,
path picture={\fill (path picture bounding box.south west) 
rectangle  ([xshift=\xfill*1em]path picture bounding box.north west);}]{};
}
\end{tikzpicture}}
%%%% Machinery to draw the rating stars

\usepackage{fancyhdr}
\thispagestyle{fancy}
\pagestyle{fancy}

\usepackage{paralist}

\usepackage{titlesec}
\titleformat{\section}{\normalfont\fontsize{12}{15}\bfseries}{\thesection}{1em}{}

\usepackage[backend=bibtex8,style=numeric]{biblatex}
\addbibresource{biblio.bib}

\usepackage[english]{babel}
\usepackage{blindtext}

\renewcommand{\thesection}{\arabic{section}}

%%%% Related Work environments
\newcounter{RelatedWorkCounter}
\addtocounter{RelatedWorkCounter}{1}
\newcommand{\relatedwork}[3]{%
\paragraph{Paper:}\fullcite{#1}
\begin{compactdesc}
\item[- How:] #2
\item[- Why:] #3
\end{compactdesc}
\stepcounter{RelatedWorkCounter}
}

%%%% Critical question environments
\newcounter{QuestionCounter}
\addtocounter{QuestionCounter}{1}
\makeatletter
\newcommand{\criticalquestion}[1]{\def\criticalquestion@required{#1}\criticalquestion@opt}
\newcommand{\criticalquestion@opt}[1]{%
\paragraph{Q\theQuestionCounter: \criticalquestion@required}
#1%
\stepcounter{QuestionCounter}
}
\makeatother

%%%%%%%%%%%%%%%%%%%%%%%%%%%%%%%%%%%%%%%%%%
% Meta Data:
%%%%%%%%%%%%%%%%%%%%%%%%%%%%%%%%%%%%%%%%%%

\lhead{Vuong Nguyen}
\rhead{Topic: Advanced Search/Paper 03}
\title{Testing Vision-Based Control Systems Using Learnable Evolutionary Algorithms}

\begin{document}
\begin{center}
\textbf{\thetitle}
\end{center}

%%%%%%%%%
% If your text is too long and you need to choose what part to cut down between Summary and Critical Discussion, always cut down on the Summary! We all have read the paper, so the really interesting part is your opinion!
% follow the lane
% regain its center on the lane
% drive within a lane center
% avoid land departure
% to depart the road
% to move away from the center of the lane
% cause the self-driving car to break out of the land bounds
% cause the ego-car to drive away from the lane center
% leads the ego-car to drive out of the road

% be amenable for
% be responsible for
%%%%%%%%%
\section{Summary}
%1/2 Page
The paper discusses the testing challenges of autonomous vehicles in real-life situations since such testing settings are expensive and dangerous. 
%
Despite having large and multidimensional input spaces, simulation-based testing of vision-based control systems such as Advanced Driver Assistance Systems (ADAS) are still in use as an alternative to the aforementioned approach.
%
And thus, the authors proposed an automated testing algorithm using learnable evolutionary algorithms to address these problems, called NSGAII-DT.
%

The approach consists of the Multi-objective Search Algorithm and Decision Tree Model intended to guide the search for the critical test scenarios within a time budget and identify the critical regions of the ADAS input space so that engineers can detect scenarios that may expose failures. 
%
To elaborate, the approach based on the NSGA-II algorithm provides a set of ADAS critical test scenarios and regions forming a Pareto non-dominated front.
%
Those test scenarios define a set of feature vectors as input and the outcome will contribute to computing fitness functions that systematize critical behaviors.
%
Furthermore, the research proposes the use of Genetic operators and
%
a combination between Boolean functions and Classification Decision Trees to perform sets of iteration on labeled test cases and create a new partition with labeled scenarios as a majority.
% 
To indicate that the new algorithm is superior in comparison to the NSGA-II algorithm, researchers validated their research by addressing two research questions.
%
The former indicates whether NSGAII-DT is more effective than the normal NSGA-II algorithm in producing critical test scenarios while the latter discusses whether the contribution of the decision tree will be able to distinguish critical regions in ADAS input spaces.
%
Regarding evaluation metrics, those quality indicators are Hypervolume (HV), Generational Distance (GD), and Spread (SP) for the comparison of the Pareto fronts between NSGAII-DT and NSGAII and 
% 
the Wilcoxon Rank Sum test and Vargha-Delaney for measuring the statistical test results.
%
Researchers executed NSGAII-DT and NSGAII to the AEB case study for 20 times within 24 hours.
%
As a result, the performance of the NSGAII-DT exceeded the NSGAII significantly and managed to improve on the number of critical test scenarios by 78\%.
% 
Also, it can generate smaller, more comparable, and more precise critical regions.



\section{Critical Discussion}
%1/2 Page
First of all, the paper was well written. In the beginning, it not only identified problems that the traditional testing framework faced in self-driving cars but also proposed an alternative solution to address those problems successfully.
%
Furthermore, researchers attempted to make their paper persuasive since they provide compelling evidence to support their arguments on why their approach statistically outperformed a baseline algorithm by answering two research questions with experimental studies.  
%
Finally, the goals of the experiment are identified and measured carefully by the authors.
%
To elaborate, the description of objective measures such as Multi-objective search and Decision tree learning is transparent throughout the paper. 
%
The results then were measured by evaluation metrics such as Hypervolume (HV), Generational Distance (GD), and Spread (SP), the Wilcoxon Rank Sum test, and Vargha-Delaney, so that they are reliable and adequate to justify the hypothesis.
%

On the other hand, the papers organized interviews with few engineers to prove the approach's advantages but the number of people might not be substantial to claim the usefulness of the approach under the statistical point of view.
%
Besides, different configurations can lead to different result analysis. Therefore, the research should add more case studies besides AEB to describe the effects of critical behaviors and scenarios after changing typical parameters.
%
Also, the AEB system is one of many fundamental problems of Automotive Software Systems. 
%
The researchers have constructed this case study along with numerous parameter variables and complex protocol whereas they miss mentioning the ability to reuse those things on other problems in the same domain such as autonomous parking.
%
Therefore, it might be time-consuming and require lots of efforts in not only understanding the approach and learning how to apply it to a new problem.
%
Last but not least, the paper introduces a lot of concepts without proper correlation. 
%
Thus, it may lead to an increase in the complexity of the paper which makes it incomprehensible, especially for people with minimal software testing experience. 
%
To conclude, I’m interested in the combination of NSGA-II and Decision Tree Model is by far an ideal, pragmatic approach for generating effective critical test cases to expose problems of the self-driving car but it might be tough to translate the concept into applicable use.
\newpage 
% Paper rating, critical questions and related work sections must always appear on the second page of the summary

\section{Rating}
% Add here the overall rating of the paper (1 start is BAD, 5 starts is VERY GOOD). Please explain in one or two sentences the reason of your evaluation and whether you suggests the paper for the next edition of the seminar.
\Stars{5}

The paper has defined many  concepts transparently and illustrated the empirical evaluation carefully with proper research questions and statistical methods. Even though there are still small problems as mentioned above, so I would rate this paper 3 stars.

Of course I highly recommend this paper for the next semester !

\section{Critical Questions}
% A least 2 questions here. If possible try to answer them or write down 
\criticalquestion{What does it mean when authors argue using decision tree is better than SVM and other Machine Learning techniques due to its understandable boundaries? }% Optional answer follow
%{Those are some important notes about the question or a possible answer to it.}

\criticalquestion{The authors suggest an upper threshold to control the number of vectors in each tree leaf not below a certain lower threshold. How does it prevent overfitting?}% Optional answer follow
% {Those are some important notes about the question or a possible answer to it.}

\section{Related Work}
% Remember that you MUST list at least 4 related work here ! Fill the bib file will all the required information and build your bibliography before submitting the paper !
\paragraph{How many other papers did you considered during for the related work?}
4


% FIRST
\relatedwork%
% Put the citation key corresponding to the paper you selected here:
{chen2018beyond}
% Explain how did you find the paper here (check the slides to see how you can effectively find related work papers)
{Searched in Google Scholar with keywords Beyond Evolutionary algorithms and Search-based Software Engineering}
% Explain why this paper is related here (do not just say, it has the same content or a similar title...)
{The paper presents an approach to address computer-intensive problem of evolutionary methods by using fewer solutions.}

% SECOND
\relatedwork
%% Put the cite key corresponding to the paper here:
{campos2017empirical}
%% Explain how did you find the paper here:
{Searched in Google Scholar with keywords Evolutionary Algorithms and Test Suite Generation.}
%% Explain why this paper is related here:
{The paper demonstrates a comparison between evolutionary algorithms and random testing to find the most effective one for test generation.}

% THIRD
\relatedwork
%% Put the cite key corresponding to the paper here:
{buhler2004automatic}
%% Explain how did you find the paper here:
{Searched in Google Scholar with keywords Autonomous Parking and Evolutionary Algorithm.}
%% Explain why this paper is related here:
{The paper describes an algorithm that define suitable fitness functions to evaluate the quality of parking.}

% FOURTH - AND PROBABLY LAST 
\relatedwork
%% Put the cite key corresponding to the paper here:
{khosrowjerdi2017learning}
%% Explain how did you find the paper here:
{Searched in Google Scholar with keywords Critical Test Cases, Autonomous Vehicles.}
%% Explain why this paper is related here:
{The paper provides a method by applying machine learning and model-checking techniques for test case generation.}


\end{document}